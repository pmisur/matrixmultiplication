\documentclass[a4paper, 12pt]{article}
\usepackage{graphicx}
\usepackage[utf8]{inputenc}
\usepackage[magyar]{babel}
\usepackage{caption}
% horizontal line
\usepackage{hhline}

% eps formátumú ábrák --> pdflatex fordításhoz!!
\usepackage{epsfig}


% ha betűszíneket is szeretnénk használni
\usepackage{color}

% Margók egyéni beállításai
\usepackage{anysize}
\marginsize{1.64cm}{1.64cm}{1.2cm}{2.4cm} %\left right top bottom

% vakszöveg
\usepackage{lipsum}
\usepackage{blindtext}

% A HIVATKOZÁSOKHOZ HASZNÁLT CSOMAGOK. RÉSZLETESEBBEN LD: google -> latex bibtex
\usepackage[numbers, square, comma, sort&compress]{natbib}
%\usepackage[format=hang,labelsep=period]{caption}
\usepackage{hyperref}
\usepackage[unicode]{hyperref}   % ezzel a hivatkozások linkké válnak
\usepackage{bookmark}
\hypersetup{bookmarksopen={true}}
\hypersetup{bookmarksopenlevel={2}}
\hypersetup{bookmarksnumbered={true}}
\hypersetup{
  colorlinks,%
  citecolor=red,%
  filecolor=black,%                                                                                                                                               
  linkcolor=blue,%
  urlcolor=green
}


\frenchspacing
\setlength{\parskip}{2ex}
\setlength{\headsep}{0,4cm}
\setlength{\headheight}{4pt}

\renewcommand{\contentsname}{Tartalomjegyzék}
%\renewcommand{\headrulewidth}{0,05pt}
%\renewcommand{\footrulewidth}{0pt}

\begin{document}

%\begin{titlepage}   
\begin{center}
\thispagestyle{empty}  

\vspace*{0.7cm}
\rule{\linewidth}{0.5mm} \\[3mm]
\vspace*{0.7cm}

{\LARGE A Kutatómunka információs eszközei}

\vspace*{0.7cm}
\rule{\linewidth}{0.5mm} \\[3mm]
\rule{\linewidth}{0.5mm} \\[3mm]



{\Large Gyakorlat\\}

\vspace*{0.7cm}
\rule{\linewidth}{0.5mm} \\[3mm]
  {Jegyzőkönyv} \\[3mm]
  \vspace*{1cm}
{\footnotesize Írta: Misur Patricia} \\
{\tiny misurpatricia@gmail.hu}

  \vspace*{2cm}

\begin{figure}[h!]
\begin{center}
\includegraphics[width=0.4\textwidth]{elte.png}
\end{center}
\end{figure}
\end{center}
\end{titlepage}


\newpage 
\tableofcontents
\newpage



\section{Bevezetés}
A jegyzőkönyvet a C programozás nevű kurzus egyik feladata során megírt kódomról írom, ami két mátrix összeszorzását valósítja meg. A program Linux operációs rendszeren (Ubuntu), C nyelven készült, a fejlesztéséhez Visual Studio Code-ot használtam, Cmake-kel fordítva. A jegyzőkönyvet \LaTeX\-ben írtam.

\subsection{A \LaTeX\-ről néhány szó}

A \LaTeX\ egy olyan szövegformázó rendszer, amely főként olyan dokumentumok létrehozására alkalmas, ami számos matematikai képletet, szimbólumot tartalmaz, ezért jelentős részben tudományos cikkek/szakdolgozatok megírásakor használják.\\*
A szöveget jelölésekkel látjuk el, amelyek hordozzák a megjelenítési információkat, majd egy fordítóprogram létrehozza a kész dokumentumot.\\*
Több operációs rendszeren elérhető, én Linuxon a Texmaker programot használom, ami a következő parancs terminálba való bevitelével telepíthető:
\begin{verbatim}
sudo apt-get install texmaker
\end{verbatim}

\begin{figure}[h!]
\begin{center}
\includegraphics[width=0.9\textwidth]{latex.jpg}
\caption*{Szép matematikai kifejezések \LaTeX\-ban írva, forrás: https://i.imgur.com/LoIcQ.jpg }
\end{center}
\end{figure}


\section{A kód fejlesztése}
\subsection{Első fázis}

A jegyzőkönyv alapjául szolgáló, C nyelvben megírt, mátrixokat összeszorzó program fejlődését mutatom be. A kódot Vscode-ban írom, a fordítást és futtatást először az operációs rendszer konzolján hajtom végre, terminálból. Elsőkörben a GitHub-ra egy olyan verziót töltöttem fel, ami "core dumbed" hibaüzenettel tért vissza. Hibás memóriakezelésnél (nem inicializált, vagy nem érvényes pointer) gyakran "Segmentation Fault (core dumped)" üzenetet kapunk. \\*
A program bemenetként 4 adatot vár, az első kettő két, külön beolvasott .txt kiterjesztésű file, amik megfelelően tördelve egy-egy mátrix objektumot jelölnek, az utolsó kettő, pedig az összeszorozhatóság feltételeit biztosító tulajdonságok.\\*
A main.c fordítása gcc-vel:
\begin{verbatim}
gcc main.c -o main
\end{verbatim}
Futtatás:
\begin{verbatim}
./main matrix.txt matrix2.txt 3 3
\end{verbatim}

\begin{figure}[h!]
\begin{center}
\includegraphics[width=0.7\textwidth]{cd.png}
\caption*{Hibás memóriakezelés hibaüzenete}
\end{center}
\end{figure}


\subsection{Második fázis}
Következő lépésként magát a szorzás műveletét elvégző függvényt kivettem, és így próbáltam futattni a kódot, ami szintén memóriaallokációs hibát dobott. Újra átnézve világossá vált, hogy a probléma ott volt, hogy előbb foglaltam memóriát az objektumnak, minthogy az értéket kapott volna. Ezt javítva, újból futtatva így már a program kiírja helyesen a beolvasott .txt fileokból a két mátrixot. (Még mindig gcc-vel fordítva.) 

\begin{figure}[h!]
\begin{center}
\includegraphics[width=0.7\textwidth]{readmat.png}
\caption*{A beolvasott két file megjelenítése konzolon}
\end{center}
\end{figure}


\subsection{A végleges, működő program}

Megtalálva a hibát, már nem volt nehéz megkonstruálni a teljes, működő kódot, csak hozzá kellett adni a szorzást elvégző kikommentelt kódrészletet.

\begin{figure}[h!]
\begin{center}
\includegraphics[width=0.7\textwidth]{mult.png}
\caption*{A beolvasott két file megjelenítése konzolon}
\end{center}
\end{figure}

\newpage

\section{Visual Studio Code}

A Vscode egy több száz nyelvet támogató, dinamikusan fejlődő fejlesztői környezet. Támogatja a Git-et.
\subsection{Cmake}
A Vscode-hoz bővítményként hozzá tudjuk adni a Cmake-et, mégpedig úgy, hogy az Extensions-re kattintva a keresőbe beírjuk, hogy Cmake, és a Cmake-et, valamint a Cmake Tools-t kell telepítenünk a Vscode-on belül. Ha ezzel megvagyunk, a projektünket a már a Cmake segítségével tudjuk buildelni.

\begin{figure}[h!]
\begin{center}
\includegraphics[width=0.7\textwidth]{cmake.png}
\caption*{Cmakekel való buildelés Vscodeon belül}
\end{center}
\end{figure}

\section{GitHub}
\subsection{Mi az a GitHub?}
A Git egy verziókezelő, arra szolgál, hogy fileok (programok, dokumentációk, stb) különböző verzióit kordában tartsa, elkönyvelje, tárolja és megossza. Röviden összefoglalva a Git annyit csinál, hogy amikor azt mondjuk neki (commit), akkor egy directoryról csinál magának egy helyi adatbázist a .git nevű könyvtárba. Ezekkel az adatbázisokkal \\*
\begin{itemize}
\item nyomon tudja követni, hogy mikor hogyan változott a könyvtárunk
\item vissza tudja állítani bármelyik korábbi (commit-olt) állapotát a könyvtárnak
\item szinkronizálni tud egy másik gépen levő hasonló könyvtárral, közben intelligensen átvezeti a változásokat, illetve jelez, ha nem megy neki
\end{itemize}
A szemléletbeli különbség ahhoz képest ahogy eddig éltünk az, hogy a dolgaink nem akkor vannak elmentve ha megnyomtuk az editorban a save gombot, hanem csak akkor, ha a git adatbázisunkba is bekerültek (commit). A dolgok tehát röviden a következőképp fognak kinézni: frissítjük a helyi adatbázisunkat a központi Szerver nevű gépéről (arra az esetre gondolva, ha valaki más továbbírta azóta, hogy láttuk; pull). Ezután dolgozunk a filejainkon. Ha eljutottunk egy olyan állapotba amit érdemes menteni, akkor megmondjuk melyik filet vegye figyelembe (add, vagy stage), majd mentjük az adatbázisba (commit). Végül pedig feltöltjük a módosított adatbázist a Szerver-re (push).

\subsection{A Git telepítése Linuxra}
Terminálba a következő sorokat kell bevinni:
\begin{verbatim}
sudo apt install git-all
\end{verbatim}
\subsection{A Git használata - Alapok}
A github.com-ra való regisztráció után, létre tudjuk hozni a saját repositorynkat, amit össze tudunk kötni a szamítógépünkkel klónozzással.

\begin{verbatim}
git clone [url cím]
\end{verbatim}
Van tehát egy könyvtárunk egy git adatbázissal. Ebben a könyvtárban módosítunk valamit, elmentjük ahogy szoktuk a munkánkat, viszont ettől még nem kerül be a Git adatbázisába. Ehhez először ki kell jelölnünk, hogy mely fileokat ajánljuk a Git figyelmébe (add. - minden file)

\begin{verbatim}
git add file1 file2
git add.
\end{verbatim}
Ezután committolunk, érdemes megjegyzést fűzni hozzá:

\begin{verbatim}
git commit -m "megjegyzes"
\end{verbatim}
Most a megváltozott adatbázisunkat szeretnénk visszatölteni a Szerverre. Ha klónozással szereztük az adatbázist, akkor ehhez csak annyit kell mondanunk, hogy:

\begin{verbatim}
git push
\end{verbatim}
\textit{Forrás: https://people.maths.bris.ac.uk/~mb13434/git/miagit.html}

\begin{figure}[h!]
\begin{center}
\includegraphics[width=0.7\textwidth]{gitgcc.png}
\caption*{A gcc-vel fordított file-ok github-ra töltése}
\end{center}
\end{figure}

\begin{figure}[h!]
\begin{center}
\includegraphics[width=0.7\textwidth]{github.png}
\caption*{A repository a push-olás után}
\end{center}
\end{figure}

A repository a GitHub-on a következő linken érhető el:
\begin{center}
\href{url}{https://github.com/pmisur/matrixmultiplication}
\end{center}
\end{document}
